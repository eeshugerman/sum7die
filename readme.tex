\documentclass[a4paper, 11pt]{article}
\usepackage{fullpage} % changes the margin
\usepackage{changepage}
\usepackage{amssymb,mathtools}
\usepackage{enumerate}

\setlength{\parindent}{0pt}

\begin{document}

\large\textbf{Data Challenge } \hfill \textbf{Elliott Shugerman} \\

\section{Problem}
Suppose you have a 7-sided die with sides 2, 4, 8, 16, 32, 64, and W. You decide to play the following game: roll the die repeatedly until the sum of the results exceeds a threshold $T$. Count W as 11 unless it will put you over the threshold, in which case count W as 1. Your score is the sum minus the threshold.


\begin{enumerate}[\hspace{5mm}I.]
	\item What is the mean and standard deviation of the distribution of possible scores if the threshold is 100? Give an exact answer and verify by simulation.
	\item What is the mean and standard deviation of the distribution of possible scores if the threshold is 1000? Give an exact answer and verify by simulation.
	\item What is the exact mean and standard deviation of the number of times you expect to roll the die to win the game if the threshold is 100? Give an exact answer and verify by simulation.
	\item Let the high value of W be any number. Create a visualization showing how the expected final score varies with the threshold and with the high value of W.

\end{enumerate}
 
\section{Solution}
This problem may not be solved by a brute-force approach, as the number of paths blows up quickly. 

In order to determine the distribution of scores, we will first generate the following data for each possible score: the number of roll sequences which produce the score $p$ (``paths"), and the number of rolls in each sequence $s$ (``steps"). That is, we want $p_n(s)$ for $T < n < T + 64$. 

In order to generate this function for $n > T$, we first, we first want this same function for all $n < T$. (I suppose we don't really need this data for $n < T - 64$ , but we'll find it along the way.) The rules which determine the paths to $n$ for $n < T$ depend on whether or not $n$ is greater than the ''critical point'' $n_c$.

\[ n_c = T - W_{high} + 1 \]



The critical point is the number at which, when reached in a path, W switches to 1. Equivalently, it is the highest number which may not be reached by a path that includes 1.\\

In summary, there are three regions, each with their own rules for choosing paths:

\medskip

\begin{enumerate}[\hspace{5mm}a.]

	\item $(0, n_c]$

	\item $(n_c, T]$

	\item $(T, T + 64]$

\end{enumerate}
\medskip

An algorithm is defined to generate $p_n(s)$ for $n$ in each region. The results for region $a$ are used to compute the results for $b$, the results for $a$ and $b$ are used to compute the results for $c$. These algorithms are described in section 2.2.\\

From here, we are nearly ready to calculate a final result. But one step of legwork remains -- we must take into account that the probability that a path will be traversed is not equal for all paths. When a single path branches into two, the relative probability that either child branch will be traversed is $1/2$ to the parent's 1; a second branching will create paths with probability $1/2^2$. In our case, 7 branches diverge each step of the way, so the relative probability that a path with $s$ steps will be traversed is $1/s^2$. Thus, we must scale the path vs step distributions for each score in the following manner:

\[ p_n^t(s) = (1 / s^2)p_n(s) \]

where $p_n^t$ is the (un-normalized) distribution of paths \textit{actually taken} and $p_n$ is the distribution of paths which \textit{exist} or \textit{could be taken}.\\

Finally, we can now construct our score distribution. The relative probability for each score is the sum of the corresponding $p_n^t$.

\subsection{Counting functions}

First, some notation:

\medskip

\begin{adjustwidth}{3.0em}{3.0em}
Let $[n]_g$ denote the number of paths to $n$, with gumballs drawn from $g$. $g$ is $L = \{1, 2, 4, . . . , 64\}$ or $H = \{11, 2, 4, . . . , 64\}$.

\end{adjustwidth}

\medskip

\begin{adjustwidth}{3.0em}{3.0em}
Let $[n]$ (no subscript) denote the number of paths to $[n]$, given the rules of the gumball game (i.e., the values we are looking for).

\end{adjustwidth}

\medskip

Note: The algorithms below merely count the number of paths to a number n. However, the number of paths \textit{for each path length} is required to solve the problem. The functions in the python code are essentially equivalent to those here, except instead of returning a single number, they return a distribution as a dict, with number of steps as keys and number of paths as values. These functions call each other (or lookup data computed by the others). When the dict [n] is computed as a sum, and [m] is an elelement in this sum, the value $[m](key=num\_steps)$ is added to (the accumulator for) $[n](key=num\_steps + 1)$.\\ 

\subsubsection{region a}


\[
	[n] = \sum_{i=0}^{\vert H \vert}
		\begin{cases}
			[n - H_i]	& \text{if $n - H_i > 0$} \\
			1		& \text{if $n - H_i = 0$} \\
			0		& \text{if $n - H_i < 0$}
		 \end{cases}
\]

\subsubsection{region b}


\textbf{TODO: double check against code}
\[
	[n] = \sum_{i=0}^{\vert L \vert}
		\begin{cases}
			[n - L_i]	& \text{if $T > n - L_i > n_c$} \\
			[n - H_i]	& \text{else if $n - H_i > 0$} \\
			1		& \text{if $n - H_i = 0$} \\
			0		& \text{if $n - L_i < 0$}
		 \end{cases}
\]

\subsubsection{region c}

\[
	[n] = \sum_{i=0}^{\vert L \vert}
		\begin{cases}
			[n - L_i]	& \text{if $n - L_i \leq T$} \\
			0		& \text{otherwise}
			
		 \end{cases}
\]


\end{document}

